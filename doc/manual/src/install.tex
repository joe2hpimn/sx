\chapter{Installation / Upgrade}

Skylable \SX is tested on all popular UNIX platforms, including Linux,
FreeBSD, and OS X. We try to support as many platforms as possible,
if you have troubles installing, compiling or running our software on
your platform please let us know.

\section{Minimum requirements}
The default setup described in this manual requires 2GB of RAM available
for each node. \SX can also be installed on machines with lower resources,
such as embedded ARM devices, but that requires advanced configuration not
covered by the manual.

\section{Binary packages}
The binary packages are available for all popular Linux distributions,
and this is the easiest and recommended way to install Skylable \SX.
Please visit\\
\url{http://www.skylable.com/download/sx}\\
for the up-to-date list of supported distributions and installation instructions.

\section{Source code}
In order to compile \SX from source, you will need the following packages to
be installed together with their development versions:
\begin{itemize}
    \item OpenSSL/NSS
    \item libcurl $\ge$ 7.34.0 (otherwise the embedded one will be used)
    \item zlib
    \item FUSE $\ge$ 2.7.0 (otherwise sxfs will not be compiled)
\end{itemize}
For example, on Debian run:
\begin{lstlisting}
# apt-get install libssl-dev libcurl4-openssl-dev libz-dev fuse libfuse-dev
\end{lstlisting}

\subsection{Compilation}
The software is based on autoconf, and you can just perform the standard
installation steps. The following commands install all the software in
\path{/usr/local}:
\begin{lstlisting}
$ ./configure && make
# make install
\end{lstlisting}
The rest of the manual assumes that \SX was installed from a binary
package, so some paths may be different.

\section{Upgrade existing cluster}
To take advantage of new features and improvements, it's recommended to
keep the cluster software up to date. The upgrade procedure has been
simplified and automated as much as possible to allow a smooth update
of a live cluster.
\subsection{Upgrading a single node}
It is recommended to upgrade one node at a time. In case of a problem, the
other nodes will stay unaffected and will be able to serve data to the
clients.
First install the latest version of Skylable \SX in the same way as the
previous deployment. Then run the following command:
\begin{lstlisting}
# sxsetup --upgrade
Updating sxhttpd.conf (vts)
Upgrading local node...
[sx_storage_upgrade]: Performing integrity check on /var/lib/sxserver/storage
[sx_storage_upgrade]: Integrity check completed in 0s
[sx_storage_upgrade]: Upgrading local databases, this may take a while...
[sx_storage_upgrade]: Committing changes
[sx_storage_upgrade]: Schema upgrade completed in 0s
[sx_storage_upgrade]: Successfully upgraded all DBs
[sx_storage_upgrade]: Storage closed in 1s
[upgrade_node]: Storage is up to date
Starting SX.fcgi
Starting sxhttpd
SX node started successfully
Moving remote sxsetup.conf into cluster settings...

More nodes in the cluster require upgrading.
\end{lstlisting}
% Could not upgrade the remote cluster data at this point. Please run
% sxsetup --upgrade again when all other nodes are locally upgraded and
% online. The cluster should also get upgraded automatically after the
% last node is locally upgraded.

The above is the expected output when not all of the nodes have been
updated yet. After upgrading the entire cluster the expected output is:
\begin{lstlisting}
# sxsetup --upgrade
Local node is up-to-date.
Versions:
	192.168.1.101: 2.2 (2.2)
	192.168.1.102: 2.2 (2.2)
	192.168.1.103: 2.2 (2.2)
Cluster already fully upgraded

All components of the cluster are up-to-date!
\end{lstlisting}
\textbf{IMPORTANT:} Due to substantial changes and improvements, all
nodes running \SX 1.x must be upgraded to 2.x otherwise some operations will
only work in read-only mode.
% When running \verb+sxsetup --upgrade+ on the last node of the cluster,
% it should also start the upgrade of the remote cluster data:
%\begin{lstlisting}
%# sxsetup --upgrade
%Upgrading local node...
%[...]
%Remote upgrade started. You can monitor the progress by running:
%/usr/sbin/sxadm cluster --info sx://admin@mycluster
%\end{lstlisting}
%The remote upgrade process can take some time, depending on the amount of
%data in the cluster. The progress can be monitored with
%\verb+sxadm cluster --info+ as instructed above. When it's done, running
%\verb+sxsetup --upgrade+ on any node should report the cluster is up to date:
