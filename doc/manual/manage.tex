\chapter{Cluster Management}

\section{Local node status}
You can check status of a specific node by running \path{sxserver status} on
that node:
\begin{lstlisting}
# /opt/sx/sbin/sxserver status
--- SX STATUS ---
sx.fcgi is running (PID 14394)
sxhttpd is running (PID 14407)

--- SX INFO ---
Cluster name: ^\marked{mycluster}^
Cluster port: 443
HashFS Version: SX-Storage 1.5
Cluster UUID: 01dca714-8cc9-4e26-960e-daf04892b1e2
Cluster authentication: CLUSTER/ALLNODE/ROOT/USERwBdjfz3tKcnTF2ouWIkTipreYuYjAAA
Admin key: ^\marked{0DPiKuNIrrVmD8IUCuw1hQxNqZfIkCY+oKwxi5zHSPn5y0SOi3IMawAA}^
Internal cluster protocol: SECURE
Used disk space: 17568768
Actual data size: 463872
List of nodes:
         * ec4d9d63-9fa3-4d45-838d-3e521f124ed3 ^\marked{192.168.1.101}^ (192.168.1.101) 536870912000
Storage location: /opt/sx/var/lib/sxserver/data
SSL private key: /opt/sx/etc/ssl/private/sxkey.pem
SX Logfile: /opt/sx/var/log/sxserver/sxfcgi.log
\end{lstlisting}
This gives you the information about local services and disk usage, but
also provides the admin key, which is needed for accessing the cluster
itself.

\section{Administrator access}
During cluster deployment a default admin account gets created
and initialized. You should be able to access the cluster from
any node using \path{sx://admin@mycluster} profile. In order
to manage the cluster remotely or from another system account,
you need to initialize access to the cluster using \path{sxinit}. 
In the example below we use the default admin account created
during cluster setup. Since "mycluster" is not a DNS name, we need
to point sxinit to one of the nodes of the cluster. It will
automatically discover the IP addresses of the other nodes.
Additionally, we create an alias \path{@cladm}, which later
can be used instead of \path{sx://admin@mycluster}.
\begin{lstlisting}
$ sxinit -l 192.168.1.101 -A @cladm sx://admin@mycluster
Warning: self-signed certificate:

        Subject: C=GB, ST=UK, O=SX, CN=mycluster
	Issuer: C=GB, ST=UK, O=SX, CN=mycluster
	SHA1 Fingerprint: 84:EF:39:80:1E:28:9C:4A:C8:80:E6:56:57:A4:CD:64:2E:23:99:7A

Do you trust this SSL certificate? [y/N] ^\marked{y}^
Trusting self-signed certificate
Please enter the user key:
^\marked{0DPiKuNIrrVmD8IUCuw1hQxNqZfIkCY+oKwxi5zHSPn5y0SOi3IMawAA}^
\end{lstlisting}

\section{User management}
\SX similarly to UNIX systems supports two types of users: regular and
administrators. A new cluster has only a single 'admin' user enabled by
default. The administrators can perform all cluster operations and access
all data in the cluster, while the regular users can only work with volumes
they have access to. It is recommended to only use the admin account for
administrative purposes and perform regular operations as a normal user.
Use \path{sxacl useradd} to add new users to the cluster:
\begin{lstlisting}
$ sxacl useradd joe @cladm
User successfully created!
Name: joe
Key : FqmlTd9CWZUuPBGMdjE46DaT1/3kx+EYbahlrhcdVpy/9ePfrtWCIgAA
Type: normal

Run 'sxinit sx://joe@mycluster' to start using the cluster as user 'joe'.
\end{lstlisting}
By default a regular user account gets created. In order to list existing
users run:
\begin{lstlisting}
$ sxacl userlist @cladm
admin (admin)
joe (normal)
\end{lstlisting}
To retrieve the current authentication key for a specific user run:
\begin{lstlisting}
$ sxacl usergetkey joe @cladm
5tJdVr+RSpA/IPuFeSwUeePtKdbDLWUKqoaoZLkmCcXTw5qzPg5e7AAA
\end{lstlisting}
Finally, to permanently delete a user from the cluster run the following
command:
\begin{lstlisting}
$ sxacl userdel joe @cladm
User 'joe' successfully removed.
\end{lstlisting}
All volumes owned by the user will be reassigned to the cluster administrator performing the removal.

\section{Volume management}
Volumes are logical partitions of the \SX storage of a specific size and accessible
by a particular group of users. Additionally, the volumes can be connected with client
side filters to perform additional operations, such as compression or encryption.
Only cluster administrators can create and remove volumes.

\subsection{Creating a plain volume}
Below we create a basic volume of size 50G owned by the user 'joe' and fully replicated on two nodes.
\begin{lstlisting}
$ sxvol create -o joe -r 2 -s 50G @cladm/vol-joe
Volume 'vol-joe' (replica: 2, size: 50G, max-revisions: 1) created.
\end{lstlisting}
By default, a volume will only keep a single revision of each file (\path{max-revisions}
parameter set to 1). The revisions are previous versions of the file stored when the file
gets modified. For example, when a volume gets created with \path{max-revisions} set to
3, and some file gets modified multiple times, then the latest 3 versions of the file will
be preserved. All revisions are accounted for their size. See FIXME for more information on
how to manage file revisions.

\subsection{Creating a filtered volume}
Filters are client side plugins, which perform operations on files or their contents, before
and after they get transferred from the \SX cluster. When a filter gets assigned to a volume,
all remote clients will be required to have that filter installed in order to access the volume.
Run the following command to list the available filters:
\begin{lstlisting}
$ sxvol filter --list
Name            Ver     Type        Short description
----            ---     ----        -----------------
undelete        1.1     generic     Backup removed files
zcomp           1.0     compress    Zlib Compression Filter
aes256          1.4     crypt	    Encrypt data using AES-256-CBC-HMAC-512
attribs         1.1     generic     File Attributes
\end{lstlisting}
We will create an encrypted volume for user 'joe'. To obtain more information
about the \path{aes256} filter run:
\begin{lstlisting}
$ sxvol filter -i aes256
'aes256' filter details:
Short description: Encrypt data using AES-256-CBC-HMAC-512 mode.
Summary: The filter automatically encrypts and decrypts all data using
	 OpenSSL's AES-256 in CBC-HMAC-512 mode.
Options: 
	nogenkey (don't generate a key file when creating a volume)
	paranoid (don't use key files at all - always ask for a password)
	salt:HEX (force given salt, HEX must be 32 chars long)
UUID: 35a5404d-1513-4009-904c-6ee5b0cd8634
Type: crypt
Version: 1.4
\end{lstlisting}
By default, the \path{aes256} filter asks for the password during volume
creation. Since we're creating a volume for another user, we pass the
\path{nogenkey} option, which delays the key creation till the first data
transfer.
\begin{lstlisting}
$ sxvol create -o joe -r 2 -s 50G -f aes256=nogenkey @cladm/vol-joe-aes
Volume 'vol-joe-aes' (replica: 2, size: 50G, max-revisions: 1) created.
\end{lstlisting}

\subsection{Listing volumes}
To get a list of all volumes in the cluster run \path{sxls} with the cluster
argument as an administrator. When the same command is run by a normal user,
it will list all volumes, which the user has access to.
\begin{lstlisting}
$ sxls -l @cladm
  VOL  r:2  -        0  53687091200   0% sx://admin@mycluster/vol-joe
  VOL  r:2  aes256   0  53687091200   0% sx://admin@mycluster/vol-joe-aes
\end{lstlisting}
When the \path{-l (--long-format)} flag is used, the command also provides
information about the volume settings and the current space usage.
